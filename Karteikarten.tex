\documentclass[a7paper,print,grid=both]{kartei}

\usepackage[english,ngerman]{babel}
\usepackage[utf8]{inputenc}
\usepackage{amsmath}
\usepackage{amssymb}
\usepackage{listings}

% This reduces the equation width by reducing the space between symbols in an equation
\medmuskip=0mu
\thinmuskip=0mu
\thickmuskip=0mu

% This creates the command used for empty words in turing machines. Copied from lecture sources.
\newcommand{\blank}{\text{\textvisiblespace}}

%\renewcommand{\cardtextstylef}{}
%\renewcommand{\cardtextstyleb}{}

\begin{document}
\begin{karte}{DTM\\Deterministische Turing-Maschine}
$M=(Q,\Sigma,\Gamma,\delta,q_0,F)$\\$Q$...Zustandsmenge $\Sigma$...Eingabealphabet\\$\Gamma$...Bandalphabet mit $\Gamma\subseteq\Sigma\cup\{\blank\}$\\$\delta$...Übergangsfkt. $Q\times\Gamma\rightarrow Q\times\Gamma\times\{L,R,N\}$\\$q_0$...Startzustand $q_0\in Q$\\$F$...akzeptierende Endzustände $F \subseteq Q$
\end{karte}
\begin{karte}{NTM\\Nichtdeterministische Turing-Maschine}
$M=(Q,\Sigma,\Gamma,\delta,q_0,F)$\\$Q$...Zustandsmenge $\Sigma$...Eingabealphabet\\$\Gamma$...Bandalphabet mit $\Gamma\subseteq\Sigma\cup\{\blank\}$\\$\delta$...Übergangsfkt. $Q\times\Gamma\rightarrow 2^{Q\times\Gamma\times\{L,R,N\}}$\\$q_0$...Startzustand $q_0\in Q$\\$F$...akzeptierende Endzustände $F \subseteq Q$
\end{karte}
\begin{karte}{Entscheidungsproblem}
Frage nach Entscheidbarkeit
\end{karte}
\begin{karte}{(Un-)Entscheidbarkeit}
Ob allen Elementen einer Menge eine Eigenschaft eindeutig nachgewiesen (bzw das Gegenteil nachgewiesen) werden kann.
\end{karte}
\begin{karte}{Semi-Entscheidbarkeit}
Ob den Elementen einer Menge, die die Eigenschaft haben, die Eigenschaft eindeutig nachgewiesen werden kann.
\end{karte}
\begin{karte}{Co-Semi-Entscheidbarkeit}
Ob den Elementen einer Menge, die die Eigenschaft nicht haben, das Gegenteil der Eigenschaft eindeutig nachgewiesen werden kann.
\end{karte}
\begin{karte}{Aufzählbarkeit}
Eigenschaft einer Menge, dass es eine "Generatorfunktion" gibt, die alle Elemente aufzählt
\end{karte}
\begin{karte}{Abzählbarkeit}
Menge, die die gleiche Mächtigkeit wie $\mathbb{N}$ hat (eindimensional unendlich bzw abzählbar unendlich)
\end{karte}
\begin{karte}{Überabzählbarkeit}
Eigenschaft einer Menge, nicht abzählbar zu sein (keine Bijektion auf $\mathbb{N}$)
\end{karte}
\begin{karte}{Halteproblem}
Frage, ob eine Maschine (zB eine TM) auf einer bestimmten Eingabe hält (oder in eine Endlosschleife geht). Ist unentscheidbar (semi-, nicht co-semi-), NP-hart
\end{karte}
\begin{karte}{Cantor-Funktion}
Die Verteilungsfunktion der Cantorverteilung
\end{karte}
\begin{karte}{Cantor-Diagonalisierung}
Bezeichung der von Cantor entwickelten Diagonalverfahren
\end{karte}
\begin{karte}{Cantors erstes Diagonalargument}
Die Mächtigkeit zweier Mengen A und B ist genau gleich, wenn  eine Bijektion zwischen A und B gibt
\end{karte}
\begin{karte}{Cantors zweites Diagonalargument}
sei $r_i$: $r_1=0,b_{11}b_{12}b_{13}...$\\$r_1=0,b_{21}b_{22}b_{23}...$\\$r_1=0,b_{31}b_{32}b_{33}...$\\$\bar{r}=0,\bar{r}_{11}\bar{r}_{22}\bar{r}_{33}...$\\$\bar{r}$ ist dann nicht in der Menge von $r_i$
\end{karte}
\begin{karte}{Cantorsche Paarungsfunktion}
Basiert auf dem Diagonalargument von Cantor $(\mathbb{N}\times\mathbb{N}\to\mathbb{N})$
\end{karte}
\begin{karte}{Ackermannfunktion}
Funktion der Form: $\varphi(a, b, 0)=a+b\,$\\$\varphi(a, 0, n+1)=\alpha(a, n)\,$\\$\varphi(a, b+1, n+1)=\varphi(a, \varphi(a, b, n+1), n)\,$ oder ähnlich mit extrem schnellem Wachstum
\end{karte}
\begin{karte}{Topologie}
\[tbd\]
\end{karte}
\begin{karte}{Gödelsche Unvollständigkeitssätze}
Die Gödelschen Unvollständigkeitssätze weisen nach das es in hinreichend starken Systemen, Aussagen geben muss die man weder formal beweisen noch widerlegen kann. Es gibt den ersten und den 2. Unvollständigkeitssatz
\end{karte}
\begin{karte}{LOOP-Programm: Definition}
P ist LOOP Programm, wenn von der Form:\\$x_i:=x_j+n$,\\$x_i:=x_j-n$,\\$LOOP x_i DO P_j END$,\\$p_i;p_j$
\end{karte}
\begin{karte}{LOOP-Programm: ADD-Funktion}
\begin{lstlisting}[mathescape=true]
ADD $x_1$ $x_2$:
$x_0$ := $x_1$ + 0;
LOOP $x_2$ DO $x_0$ = $x_0$ + 1 END
\end{lstlisting}
\end{karte}
\begin{karte}{LOOP-Programm: SUB-Funktion}
\begin{lstlisting}[mathescape=true]
SUB $x_1$ $x_2$:
$x_0$ := $x_1$ + 0;
LOOP $x_2$ DO $x_0$ = $x_0$ - 1 END
\end{lstlisting}
\end{karte}
\begin{karte}{LOOP-Programm: MUL-Funktion}
\begin{lstlisting}[mathescape=true]
MUL $x_1$ $x_2$:
$x_0$ := $x_1$ + 0;
LOOP $x_2$ DO ADD $x_0$ $x_1$ END
\end{lstlisting}
\end{karte}
\begin{karte}{LOOP-Programm: POT-Funktion}
\begin{lstlisting}[mathescape=true]
POT $x_1$ $x_2$:
$x_0$ := $x_1$ + 0;
LOOP $x_2$ DO MUL $x_0$ $x_1$ END
\end{lstlisting}
\end{karte}
\begin{karte}{LOOP-Programm: DIV-Funktion}
\begin{lstlisting}[mathescape=true]
\end{lstlisting}
\[tbd\]
\end{karte}
\begin{karte}{LOOP-Programm: MAX-Funktion}
\begin{lstlisting}[mathescape=true]
MAX $x_1$ $x_2$:
$x_0$ := $x_1$ + 0;
SUB $x_0$ $x_2$;
ADD $x_0$ $x_2$
\end{lstlisting}
\end{karte}
\begin{karte}{LOOP-Programm: MIN-Funktion}
\begin{lstlisting}[mathescape=true]
MIN $x_1$ $x_2$:
$x_0$ = $x_1$ + 0;
MAX $x_1$ $x_2$;
ADD $x_0$ $x_2$;
SUB $x_0$ $x_1$
\end{lstlisting}
\end{karte}
\begin{karte}{LOOP-Programm: MOD-Funktion}
\begin{lstlisting}[mathescape=true]
MOD $x_1$ $x_2$:
LOOP $x_2$ DO:
    LOOP $x_1$ DO $x_0$ = $x_1$ + 0 END;
    SUB $x_1$ $x_2$
END
\end{lstlisting}
\end{karte}
\begin{karte}{LOOP-Programm: GGT-Funktion}
\begin{lstlisting}[mathescape=true]
GGT $x_1$ $x_2$:
$x_4$ = $x_1$ + 0;
LOOP $x_4$ DO:
    LOOP $x_2$ DO:
        $x_5$ = $x_2$ + 0;
        MOD $x_5$ $x_1$;
        $x_1$ = $x_2$ + 0
    END;
    $x_2$ = $x_5$ + 0
END;
$x_0$ = $x_1$
\end{lstlisting}
\end{karte}
\begin{karte}{LOOP-Programm: Fallunterscheidung}
\begin{lstlisting}[mathescape=true]
IF $x_0$ != 0 THEN P END:
    LOOP $x_0$ DO $x_1$ := 1 END;
    LOOP $x_1$ DO P END
\end{lstlisting}
\end{karte}
\begin{karte}{WHILE-Programm}
\begin{lstlisting}[mathescape=true]
P ::= x_i := x_j + c
P ::= x_i := x_j - c
P ::= P; P
P ::= LOOP x_i DO P END
P ::= WHILE x_i \neq 0 DO P END
\end{lstlisting}
\end{karte}
\begin{karte}{Kolmogorov-Komplexität}
Maß für die Strukturiertheit einer Zeichenkette, gegeben durch die Länge des kürzesten Programms, das diese Zeichenkette erzeugt.
\end{karte}
\begin{karte}{Many-One-Reduktion}
Problem $A$ ist auf $B$ many-one-reduzierbar ($A \leq_m B$), falls es eine berechenbare Funktion $f:A\rightarrow B$ gibt.
\end{karte}
\begin{karte}{Schubfachprinzip}
Falls man $n$ Objekte auf $m$ Mengen ($n,m>0$) verteilt und $n>m$ gilt, gibt es mindestens eine Menge, die mehr als 1 Objekt enthält. Auch: Taubenschlagprinzip, Dirichlet-Prinzip.
\end{karte}
\begin{karte}{Satz von Rice}
Es ist unmöglich, eine beliebige, nicht-triviale Eigenschaft der erzeugten Funktion einer Turing-Maschine algorithmisch zu entscheiden. Trivial wäre "immer akzeptieren" oder "immer verwerfen".
\end{karte}
\begin{karte}{PKP oder PCP\\Postsches Korrespondenzproblem}
Beispiel für ein unentscheidbares Problem.
\end{karte}
\begin{karte}{Äquivalenzproblem}
\[tbd\]
\end{karte}
\begin{karte}{P, NP, coNP, PSPACE}
\[tbd\]
\end{karte}
\begin{karte}{(P,NP,PSPACE)-hart}
\[tbd\]
\end{karte}
\begin{karte}{(P,NP,PSPACE)-vollständig}
\[tbd\]
\end{karte}
\begin{karte}{Wortproblem Deterministischer Endlicher Automaten}
\[tbd\]
\end{karte}
\begin{karte}{SAT\\Erfüllbarkeitsproblem}
Entscheidungsproblem, ob eine aussagenlogische Formel erfüllbar ist
\end{karte}
\begin{karte}{Kleene-Stern}
\[tbd\]
\end{karte}
\begin{karte}{Liste von P-vollständigen Problemen}
\[tbd\]
\end{karte}
\begin{karte}{Liste von NP-vollständigen Problemen}
\[tbd\]
\end{karte}
\begin{karte}{Formalisieren (Ablauf)}
\[tbd\]
\end{karte}
\begin{karte}{3SAT}
\[tbd\]
\end{karte}
\begin{karte}{QBF}
\[tbd\]
\end{karte}
\begin{karte}{LBA\\Linear Bounded Automaton}
\[tbd\]
\end{karte}
\begin{karte}{Pränexform}
\[tbd\]
\end{karte}
\begin{karte}{Skolemform}
\[tbd\]
\end{karte}
\begin{karte}{Klauselform}
\[tbd\]
\end{karte}
\begin{karte}{$\models$}
\[tbd\]
\end{karte}
\begin{karte}{Resolutionsverfahren}
\[tbd\]
\end{karte}
\begin{karte}{Unifikator}
\[tbd\]
\end{karte}
\begin{karte}{Allgemeinster Unifikator}
\[tbd\]
\end{karte}
\begin{karte}{Herbrand-Universum}
\[tbd\]
\end{karte}
\begin{karte}{Herbrand-Modell}
\[tbd\]
\end{karte}
\begin{karte}{Herbrand-Expansion}
\[tbd\]
\end{karte}
\end{document}
