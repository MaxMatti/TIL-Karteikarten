\documentclass[frontgrid,backgrid]{flacards}

\usepackage[english,ngerman]{babel}
\usepackage[utf8]{inputenc}
\usepackage{amsmath}

\begin{document}
\card{DTM\\Deterministische Turing-Maschine}{\[tbd\]}
\card{NTM\\Nichtdeterministische Turing-Maschine}{\[tbd\]}
\card{Entscheidungsproblem}{\[tbd\]}
\card{(Un-)Entscheidbarkeit}{\[tbd\]}
\card{Aufzählbarkeit}{\[tbd\]}
\card{Abzählbarkeit}{\[tbd\]}
\card{Überabzählbarkeit}{\[tbd\]}
\card{Halteproblem}{\[tbd\]}
\card{Cantor-Funktion}{\[tbd\]}
\card{Cantor-Diagonalisierung}{\[tbd\]}
\card{Cantors erstes Diagonalargument}{\[tbd\]}
\card{Cantors zweites Diagonalargument}{\[tbd\]}
\card{Cantorsche Paarungsfunktion}{\[tbd\]}
\card{Ackermannfunktion}{\[tbd\]}
\card{Topologie}{\[tbd\]}
\card{Gödelsche unvollständigkeitssätze}{\[tbd\]}
\card{LOOP-Programm: Definition}{\[tbd\]}
\card{LOOP-Programm: ADD-Funktion}{
	\begin{equation*}
	\begin{aligned}
ADD x_1 x_2:\\
x_0 := x_1 + 0;\\
LOOP x_2 DO x_0 = x_0 + 1 END
	\end{aligned}
	\end{equation*}
}
\card{LOOP-Programm: SUB-Funktion}{
	\begin{equation*}
	\begin{aligned}
SUB x_1 x_2:\\
x_0 := x_1 + 0;\\
LOOP x_2 DO x_0 = x_0 - 1 END
	\end{aligned}
	\end{equation*}
}
\card{LOOP-Programm: MUL-Funktion}{
	\begin{equation*}
	\begin{aligned}
MUL x_1 x_2:\\
x_0 := x_1 + 0;\\
LOOP x_2 DO ADD x_0 x_1 END
	\end{aligned}
	\end{equation*}
}
\card{LOOP-Programm: POT-Funktion}{
	\begin{equation*}
	\begin{aligned}
POT x_1 x_2:\\
x_0 := x_1 + 0;\\
LOOP x_2 DO MUL x_0 x_1 END
	\end{aligned}
	\end{equation*}
}
\card{LOOP-Programm: DIV-Funktion}{\[tbd\]}
\card{LOOP-Programm: MAX-Funktion}{
	\begin{equation*}
	\begin{aligned}
MAX x_1 x_2:\\
x_0 := x_1 + 0;\\
SUB x_0 x_2;\\
ADD x_0 x_2
	\end{aligned}
	\end{equation*}
}
\card{LOOP-Programm: MIN-Funktion}{
	\begin{equation*}
	\begin{aligned}
MIN x_1 x_2:\\
x_0 = x_1 + 0;\\
MAX x_1 x_2;\\
ADD x_0 x_2;\\
SUB x_0 x_1
	\end{aligned}
	\end{equation*}
}
\card{LOOP-Programm: MOD-Funktion}{
	\begin{equation*}
	\begin{aligned}
MOD x_1 x_2:\\
LOOP x_2 DO:\\
	LOOP x_1 DO x_0 = x_1 + 0 END;\\
	SUB x_1 x_2\\
END
	\end{aligned}
	\end{equation*}
}
\card{LOOP-Programm: GGT-Funktion}{
	\begin{equation*}
	\begin{aligned}
GGT x_1 x_2:\\
x_4 = x_1 + 0;\\
LOOP x_4 DO:\\
	LOOP x_2 DO:\\
		x_5 = x_2 + 0;\\
		MOD x_5 x_1;\\
		x_1 = x_2 + 0\\
	END;\\
	x_2 = x_5 + 0\\
END;\\
x_0 = x_1
	\end{aligned}
	\end{equation*}
}
\card{LOOP-Programm: Fallunterscheidung}{
	\begin{equation*}
	\begin{aligned}
IF x != 0 THEN P END:\\
	LOOP x DO y := 1 END;\\
	LOOP y DO P END\\
	\end{aligned}
	\end{equation*}
}
\card{WHILE-Programm: Definition}{\[tbd\]}
\card{WHILE-Programm: Syntax}{\[tbd\]}
\card{Kolmogorov-Komplexität}{\[tbd\]}
\card{Many-One-Reduktion}{\[tbd\]}
\card{Turing-Reduktion}{\[tbd\]}
\card{Schubfachprinzip}{\[tbd\]}
\card{Satz von Rice}{\[tbd\]}
\card{Postsches Korrespondenzproblem}{\[tbd\]}
\card{Äquivalenzproblem}{\[tbd\]}
\card{P, NP, coNP, PSPACE}{\[tbd\]}
\card{{P,NP,PSPACE}-hart}{\[tbd\]}
\card{{P,NP,PSPACE}-vollständig}{\[tbd\]}
\card{Wortproblem Deterministischer Endlicher Automaten}{\[tbd\]}
\card{Erfüllbarkeitsproblem}{\[tbd\]}
\card{Kleene-Stern}{\[tbd\]}
\card{Liste von P-vollständigen Problemen}{\[tbd\]}
\card{Liste von NP-vollständigen Problemen}{\[tbd\]}
\card{Formalisieren (Ablauf)}{\[tbd\]}
\card{SAT}{\[tbd\]}
\card{3SAT}{\[tbd\]}
\card{QBF}{\[tbd\]}
\card{LBA\\Linear Bounded Automaton}{\[tbd\]}
\card{Pränexform}{\[tbd\]}
\card{Skolemform}{\[tbd\]}
\card{Klauselform}{\[tbd\]}
\card{Herbrand-Universum}{\[tbd\]}
\card{Herbrand-Modell}{\[tbd\]}
\card{Herbrand-Expansion}{\[tbd\]}
\card{Resolutionsverfahren}{\[tbd\]}
\card{Prädikatenlogik}{\[tbd\]}
\card{Prädikatenlogik erster Stufe}{\[tbd\]}
\end{document}
